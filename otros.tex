\chapter{Software Libre en otros ámbitos}

\section{Software libre en la educación}

{\bf El software libre es importante en la educación por las siguientes razones:}
\begin{itemize}
\item El software libre supone un ahorro económico para las escuelas, pero este es un beneficio secundario. El ahorro es posible porque el software libre le da a las escuelas, igual que a cualquier otro usuario, la libertad de copiar y redistribuir el software. Así, el sistema educativo puede entregar una copia del programa a todas las escuelas, y cada una de ellas puede instalarlo en todos los ordenadores que posea sin estar obligada a pagar por ello.
\item Las escuelas tienen una misión social: enseñar a los alumnos a ser ciudadanos de una sociedad fuerte, capaz, independiente, solidaria y libre. Deben promover el uso de software libre al igual que promueven la conservación y el voto. Enseñando el software libre, las escuelas pueden formar ciudadanos preparados para vivir en una sociedad digital libre. Esto ayudará a que la sociedad entera se libere del dominio de las mega corporaciones. Enseñar el uso de un programa que no es libre equivale, por el contrario, a inculcar la dependencia, lo cual se opone a la misión social de las escuelas. Las escuelas no deben hacerlo, nunca.
\item El software libre permite a los alumnos aprender cómo funciona el software. Las escuelas que utilicen software libre contribuirán al progreso de los alumnos más brillantes en programación. Algunos alumnos son programadores natos, de adolescentes anhelan aprender absolutamente todo sobre los ordenadores y el software. Manifiestan una intensa curiosidad por leer el código fuente de los programas que usan a diario.
\item La tarea fundamental de las escuelas es enseñar a ser buenos ciudadanos, incluyendo el hábito de ayudar a los demás. En el ámbito informático, esto se traduce en enseñar a compartir el software.
\end{itemize}


{\bf ¿Por qué algunos programadores de software privativo ofrecen a las escuelas copias gratuitas de programas que no son libres?}

Porque quieren utilizar a las escuelas para imponer la dependencia de sus productos. No entregarán copias gratuitas a los estudiantes una vez que se hayan graduado, así como tampoco a las empresas para las cuales trabajarán. Una vez que uno es dependiente, se espera que pague, y las futuras actualizaciones pueden ser costosas.


{\bf Algunos sistemas operativos libres para la educación:}
\begin{itemize}
\item Skolelinux/Debian Edu.
\item Fedora Education spin.
\item OpenSUSE: Education-Li-f-e.
\end{itemize}

{\bf Software libre utilizado en la educación Argentina:}
\begin{itemize}
\item Huayra: El sistema operativo libre basado en GNU/Linux-, viene desde el 2013 instalado en las netbooks que Conectar Igualdad entrega a estudiantes secundarios de todo el país. Sistema operativo libre, se invirtió en desarrollo local y generó puestos de trabajo al mismo tiempo que se incentivó la creatividad argentina. Fue desarrollado por el Centro Nacional de Investigación y Desarrollo de Tecnologías Libres, (CENITAL), un área del programa concebida con la idea de generar y motorizar las experiencias de I D en el campo de las Tecnologías Libres.
\item Programa Conectar Igualdad: El Programa Conectar Igualdad incluye desde sus inicios software libre. A medida que se afianza esta inclusión se está haciendo mucho más enérgica. El año pasado comenzaron los cursos básicos para alumnos de Software Libre y programación. Las máquinas cuentan con más de 50 programas libres. Cuando los usuarios inician las máquinas pueden elegir usar el sistema operativo libre GNU/Linux. En él encontrarán los mismos programas o las versiones libres de programas privativos. Además, en la página de educ.ar pueden ver los programas libres que el equipo de desarrollo está adaptando.
\item El portal educ.ar: Dispone de diferentes materiales desarrollados a partir de software libre: desde juegos hasta un CD de la Colección educ.ar, con actividades específicas, como por ejemplo contenidos pensados para introducir a los docentes en el mundo del software libre: sus postulados ideológicos, teóricos, técnicos y metodológicos, acompañados por sugerencias de trabajo con programas, herramientas y modelos de esta nueva modalidad de producción, construcción y difusión de los contenidos.
\end{itemize}

\section{Software libre en organizaciones sociales}
Hay muchas razones, por sus principios filosóficos por razones jurídicas, sociales y socieconómicas, funcionalidad, accesibilidad y seguridad. Para las organizaciones y movimientos sociales de América Latina y del mundo, el software libre puede ser incorporado como herramienta tecnológica y, por qué no, también como herramienta de liberación.

{\bf Por sus principios filosóficos:}
\begin{itemize}
\item  Considera la libertad y la solidaridad como principios éticos fundamentales.
\item  Considera el conocimiento como un bien público que beneficia a la colectividad en general y permite el desarrollo igualitario.
\item  El software es conocimiento y debe difundirse sin trabas.
\item   Coloca el beneficio de la humanidad por encima de cualquier cosa.
\end{itemize}

Por razones jurídicas:
\begin{itemize}
\item  Es legal compartir y regalar software libre a otras personas.
\item   Evita los problemas de uso ilegal de licencias de software propietario/privativo y sus posibles implicaciones legales (demandas, multas, entre otros).
\end{itemize}

{\bf Por razones sociales y socioeconómicas:}
\begin{itemize}
\item Promueve la economía solidaria y el comercio justo.
\item  C Contempla a las personas usuarias como sujetos de derechos no como consumidores de productos.
\item   Se desarrolla en términos colaborativos y participativos.
\item  Las organizaciones no cuentan con muchos recursos económicos y se necesita destinar los que tiene para el logro de su misión y objetivos.
\item  Las licencias del software propietario/privativo son muy caras para las organizaciones sociales (costo de pagar licencias para cada computadora y programas 
especializados).
\item En la mayoría de los casos, las organizaciones no podrían costear las multas impuestas judicialmente como resultado de demandas legales por el uso ilegal de 
licencias de software privativo.
\item Se podría decir que el software libre es amigable con el ambiente, al no requerir de la sustitución tan rápidamente de equipo, se produce menos basura tecnológica.
\end{itemize}

{\bf Funcionalidad, accesibilidad y seguridad:}
\begin{itemize}
\item El software libre garantiza la seguridad que implica corrección de problemas de forma oportuna; en el movimiento de software libre hay equipos humanos trabajando para corregir los problemas de seguridad, adaptado a las personas, contextos e idioma.
\item   Mucha de la información que manejan las organizaciones de la sociedad civil es
información “delicada” que necesita ser resguardada, el software libre también ofrece herramientas para asegurar la información.
\end{itemize}

{\bf ¿Por qué se puede optar?}

Como sistema operativo se puede optar por GNU/Linux en cualquiera de sus tantas distribuciones, Ubuntu (www.ubuntu.org) o Debian www.debian.org).
El navegador de internet Firefox reemplaza al Internet Explorer y es lejos mucho mejor que éste. Para aplicaciones de oficina OpenOffice.org no tiene nada que envidiarle a su rival Office de Microsoft. Para edición de imágenes GIMP; para ilustraciones vectorizadas InSkape, para diseño html NVU, para FTP FilleZilla y la lista continúa.
El uso de software libre en las organizaciones y movimientos sociales, donde la autonomía y el modelo de sociedad y de desarrollo se ponen en juego, permite una alternativa al modelo dominante. El software libre como conjunto de herramientas y aplicaciones para facilitar procesos y ser útil a las luchas de los pueblos por su emancipación es indispensable en la construcción de otro mundo posible.

\section{Software libre en el Estado}
 {\bf¿Que gana el estado con el uso de software libre?}
 
Muchas veces se pone adelante de todas las ventajas el ahorro monetario. Según los sistemas instalados, sus costos, y las herramientas disponibles para reemplazarlas, este ahorro puede ser realmente importante, pero puede ser mermado a corto plazo por los costos de realizar la transición de los sistemas.
Aún así, existen muchas otras ventajas en el uso de software libre, que son inmediatas y más importantes, al punto de ser cruciales para la adopción de estas políticas por el estado:
\begin{itemize}
\item  Independencia tecnológica: Mediante el uso de software libre, el Estado deja de tener sus sistemas controlados por una entidad externa (con frecuencia empresas extranjeras). De esta forma rompe la dependencia tecnológica que lo tiene actualmente atado y obtiene las libertades que el software libre otorga.
\item  Control de la información: Esto sale directamente como consecuencia directa de las libertades del software libre. Al tener la libertad de inspeccionar el mecanismo de funcionamiento del software y la manera en que almacena los datos, y la posibilidad de modificar (o contratar a alguien que modifique) estos aspectos, queda en manos del Estado la llave del acceso a la información (en vez de quedar en manos privadas).
\item  Confiabilidad y estabilidad: El software libre, al ser público, está sometido a la inspección de una multitud de personas, que pueden buscar problemas, solucionarlos, y compartir la solución con los demás. Debido a esto, y a lo que se llama “el principio de Linus” (dada la suficiente cantidad de ojos, cualquier error del software es evidente), los programas libres gozan de un excelente nivel de confiabilidad y estabilidad, requerido para las aplicaciones críticas del Estado.
\item  Seguridad: Este es uno de los puntos clave para el Estado. Mucha de la información que el Estado maneja puede ser peligrosa en manos incorrectas. Es por esto que es crítico que el Estado pueda fiscalizar que su software no tenga puertas de entrada traseras, voluntarias o accidentales, y que pueda cerrarlas en caso de encontrarlas; tal inspección sólo es posible con el software libre.
\end{itemize}

 {\bf¿Qué problemas puede enfrentar implementar estas políticas?}
 
Obviamente, una propuesta de implementación reduciría los beneficios de los vendedores de software cerrado. Es de esperar que éstos ejerzan toda la presión a su disposición para evitar que se tomen medidas que involucren migración a software libre. Frente a ese peligro, hay que considerar que está en juego el control de la información por parte del Estado, y las libertades individuales de los ciudadanos. Debe tenerse en cuenta que una política de este tipo no discrimina en contra de productos o proveedores específicos, sino contra ciertas prácticas nocivas que involucran el control de la información del usuario por parte del proveedor. Es fundamental que el Estado no se someta a estas presiones. Eso es lo que motiva las restricciones de esta política, que tienen como fin establecer cualidades mínimas para garantizar los derechos de los ciudadanos, la calidad del software y la seguridad de la información. Toda empresa que acepte proveer sus productos sin comprometer estos derechos no tendrá problema alguno al hacerlo.


  {\bf El Software y la Seguridad Nacional:}
 
 El Estado debe almacenar y procesar información relativa a los ciudadanos. La relación entre el individuo y el Estado depende de la privacidad e integridad de estos datos, que por consiguiente deben ser adecuadamente resguardados contra tres riesgos específicos:
 
\begin{itemize}
\item  Riesgo de filtración: los datos confidenciales deben ser tratados de tal manera que el acceso a ellos sea posible exclusivamente para las personas e instituciones autorizadas.
\item   Riesgo de imposibilidad de Acceso: los datos deben ser almacenados de tal forma que el acceso a ellos por parte de las personas e instituciones autorizadas esté garantizado durante toda la vida útil de la información.
\item Riesgo de manipulación: la modificación de los datos debe estar restringida, nuevamente, a las personas e instituciones autorizadas.
\end{itemize}
La concreción de cualquiera de estas tres amenazas puede tener consecuencias graves tanto para el Estado como para el individuo. 

{\bf  El Software Libre Atiende las Necesidades de la Seguridad Nacional}

El software libre permite al usuario la inspección completa y exhaustiva del mecanismo mediante el cual procesa los datos. Sin la posibilidad de la inspección, es imposible saber si el programa cumple con su función, o si además incluye vulnerabilidades intencionales o accidentales que permitan a terceros acceder indebidamente a los datos, o impedir que los usuarios legítimos de la información puedan usarlo. 
El hecho de permitir la inspección del programa es una excelente medida de seguridad, ya que al estar expuestos los mecanismos, estos están constantemente a la vista de profesionales capacitados, con lo que se vuelve inmensamente más difícil ocultar funciones maliciosas, aún si el usuario final no se toma el trabajo de buscarlas él mismo.

{\bf Software Libre que utiliza el Gobierno}
\begin{itemize}
\item  OpenStack: Sistema capaz de realizar diversas funciones en la nube, automatización, nubes privadas y publicas, etc, infinidad de cosas que el sistema OpenStack permite realizar y que por supuesto fue nacida y creada precisamente por el gobierno.
\item Jenkins: Herramienta que es capaz de encontrar todo tipo de errores en el código fuente de un sistema. Es por eso que las empresas de gobierno utilizan tanto Jenkins. El objetivo es tener una serie de sistemas libres de errores, nada de fugas, nada de entradas externas, con Jenkins todo queda blindado y solucionado. 
\item WineHQ: Esta aplicación, lo que hace es que es capaz de emular el sistema operativo de Windows. De esta forma no importa que sistema operativo utilices, podrás hacer uso de tus programas tradicionales sin ningún problema. Te recuerdo que WineHQ también es libre, por lo que la podrás descargar fácilmente de internet.
\end{itemize}

\subsection{Voto Electrónico}

{\bf Consideraciones sobre el voto electrónico}
\begin{itemize}
\item Hay dos tipos: DREs (Direct Recording Electronic)  vs  EBPs (Electronic Ballot Printers).
\item Requerimientos Básicos del Sistema de Voto Electrónico: secreto del voto, fidelidad del voto, solo los electores autorizados pueden votar, cada lector debe emitir un solo voto.
\item No es Fácil hacer un sistema que pueda ser auditado.
\item Problema de las amenazas.
\item El acceso al código debe ser lo más abierto posible (Open source Software).
\end{itemize}

{\bf Casos fallidos en el mundo}

\begin{itemize}
\item Alemania : Los sistemas usados hasta el momento se declararon inconstitucionales.
\item Holanda : dejó de usarse en 2007 al probarse que los votos podían ser leídos a varios metros de distancia (en algunas máquinas), y que los programas podían ser alterados. EEUU múltiples errores en diversos estados.
\item India:  hackers lograron manipular los resultados con un celular. 
\item Irlanda : evaluaron un sistema en elecciones piloto y determinaron que no se podía garantizar la integridad de ninguna elección que usara ese sistema. Costo del experimento: 54 millones de euros.
\end{itemize}

\section{Licencia Creative Commons}
Poner tus obras bajo una licencia Creative Commons no significa que no tengan copyright. Este tipo de licencias ofrecen algunos derechos a otras personas bajo ciertas condiciones. ¿Qué condiciones? Esta web te permite escoger o unir las condiciones de la siguiente lista:

\begin{itemize}
\item Atribución (Attribution): En cualquier explotación de la obra autorizada por la licencia será necesario reconocer la autoría (obligatoria en todos los casos).
\item No Comercial (Non commercial): La explotación de la obra queda limitada a usos no comerciales.
\item Sin obras derivadas (No Derivate Works): La autorización para explotar la obra no incluye la posibilidad de crear una obra derivada.
\item Compartir Igual (Share alike): La explotación autorizada incluye la creación de obras derivadas siempre que mantengan la misma licencia al ser divulgadas.
\end{itemize}

Con estas condiciones se pueden generar las seis combinaciones que producen las licencias Creative Commons:

\begin{itemize}
\item Atribución (by): Se permite cualquier explotación de la obra, incluyendo la explotación con fines comerciales y la creación de obras derivadas, la distribución de las cuales también está permitida sin ninguna restricción. Esta licencia es una licencia libre según la Freedom Defined.
\item Reconocimiento – Compartir Igual (by-sa): Se permite el uso comercial de la obra y de las posibles obras derivadas, la distribución de las cuales se debe hacer con una licencia igual a la que regula la obra original. Esta licencia es una licencia libre según la Freedom Defined.
\item Atribución – No Comercial (by-nc): Se permite la generación de obras derivadas siempre que no se haga con fines comerciales. Tampoco se puede utilizar la obra original con fines comerciales. Esta licencia no es una licencia libre.
\item Atribución – No Comercial – Compartir Igual (by-nc-sa): No se permite un uso comercial de la obra original ni de las posibles obras derivadas, la distribución de las cuales se debe hacer con una licencia igual a la que regula la obra original. Esta licencia no es una licencia libre.
\item Atribución – Sin Obra Derivada (by-nd): Se permite el uso comercial de la obra pero no la generación de obras derivadas. Esta licencia no es una licencia libre.
\item Atribución – No Comercial – Sin Obra Derivada (by-nc-nd): No se permite un uso comercial de la obra original ni la generación de obras derivadas. Esta licencia no es una licencia libre, y es la más cercana al derecho de autor tradicional.
\end{itemize}

{\bf Obtener la licencia}

Cuando hayas hecho tu elección tendrás la licencia adecuada para tu trabajo expresada de tres formas:

\begin{itemize}
\item Commons Deed: Es un resumen fácilmente comprensible del texto legal con los íconos relevantes.
\item Legal Code: El código legal completo en el que se basa la licencia que elegiste.
\item Digital Code: El código digital, que puede leer la máquina y que sirve para que los motores de búsqueda y otras aplicaciones identifiquen tu trabajo y sus condiciones de uso.
\end{itemize}

{\bf Utilizar la licencia}

Una vez escogida la licencia tenés que incluir el ícono de Creative Commons “Algunos derechos reservados” en tu sitio, cerca de tu obra. Este ícono enlaza con el Commons Deed, de forma que todos puedan estar informados de las condiciones de la licencia. Si encontrás que alguien cometió una infracción a la licencia, entonces tendrás las bases para poder defender tus derechos.

{\bf Creative commons en Argentina}

Creative Commons Argentina está a cargo de dos organizaciones afiliadas: Fundación Vía Libre y Wikimedia Argentina.


