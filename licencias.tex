\documentclass{article}
% preámbulo
\usepackage[spanish]{babel}
\usepackage[utf8]{inputenc}
\title{Licencias de software}
\author{Bartoszensky, Luciano
\and
Cuello, Nahuel Ignacio
\and
Ferrero, Facundo
\and
Niño, Jeremías
\and
Segurado, Lucas Martín
\and
Zamora, Fernando
}

% cuerpo del documento
\begin{document}
\maketitle
\newpage
\tableofcontents
\newpage
\section{¿Qué es una Licencia de Software Libre?}

Una licencia de software libre es un impreso que otorga al receptor de una pieza de software derechos extensivos para modificarla y redistribuir ese software. Estas acciones normalmente se prohiben por las leyes de copyright, pero el que posee los derechos, normalmente el autor de un trozo de software, puede eliminar esas restricciones acompañando al software con una licencia de software que otorgue al receptor de estos derechos. El software que usa tales licencias se denomina software libre y esas libertades las concede el propietario del copyright.
\\
\\
{\bf Historia}
\\
Antes de 1980 no había licencias, sino que eran estatutos informales escritos por los desarrolladores, donde todavía no se escribìan las licencias para que sean defendidas en un juzgado, ya que las mismas eran compartidas por todos, y no había problemas de “adjudicación”.
No existía el “Copyleft” todavìa y eran licencias del tipo permisivas.

Entre 1990-2000 las empresas escrìbian sus propias licencias o utilizaban licencias de otros modificando determinadas restricciones para así utilizarlas bajo su nombre. 
Estas licencias produjo problemas de complejidad y compatibilidad, ya que no se podían incluir en muchos casos, módulos o librerías, o no se podía utilizar un cierto software de manera libre como se venía haciendo.

La introducción de licencias pagas, llevó a aumentar los costos en muchos casos, de los productos desarrollados, ya que para hacer uso de ciertos aplicativos, módulos y demás, se tuvo que empezar a pagar.
\newpage
\section{Organismos}
\begin{itemize}
\item {\bf Fundación para el software libre(FSF) }
Creada por Richard Stallman en 1985 con la finalidad de eliminar restricciones sobre copias, redistribución, entendimiento y modificación .Esto ayudó a promover el software libre, especialmente en el desarrollo de SO GNU. Actualmente se centra en asuntos legales, organizativos y promocionales del software libre. Stallman fue el creador del Copyleft 

\item {\bf Open Source Initiative(OSI)}
Creada por Bruce Perens y Eric S. Raymond en 1998 con la finalidad de promover el software libre. A fines del 1998 publicaron el código fuente de Netscape Communicator dadas las pérdidas y dura competencia con Internet Explorer de Windows. No podían competir ya que IE venía gratis con Windows, en cambio su producto era por separado y pago. Luego con la publicación del código fuente, hubo un grupo de personas que introdujo el término mercadotecnia que permitió levantar al software libre debido a que era más amigable. Gracias a este proyecto, surgen otros como Mozilla. 

\item {\bf Proyecto Fedora}
Comunidad responsable de la distribución de Fedora y otros proyectos. Es la fusión de Red Hat y el antiguo proyecto Fedora en 2003. El viejo Fedora desarrollaba paquetes para las viejas distribuciones de Red Hat (RHL 8 Y 9, FC1 Y 2).
\item {\bf Directrices de software libre de Debian}
Directrices que permiten ver si una licencia de software es una licencia de software libre y además si alguna licencia y software pueden incluirse en la distribución de Debian. Algunas de las directrices que podemos mencionar son:

\begin{itemize}
	\item Libre redistribución
	\item Inclusión del código fuente
	\item Permitir que las modificaciones y trabajos derivados sean hechos bajo la misma licencia
	\item	Integridad del código fuente del autor, se debe permitir cuando menos la distribución de modificaciones por medio de parches
	\item Sin discriminación de personas o grupos
	\item Sin discriminación de áreas de iniciativa, como el uso comercial
	\item Distribución de la licencia, se necesita aplicar a todo al que se redistribuya el programa
	\item La licencia no debe ser específica a Debian, básicamente reiteración del punto anterior
	\item La licencia no debe contaminar otro software
\end{itemize}
\end{itemize}
\newpage
\section {Problemas prácticos con las Licencias}
\begin{itemize}
	\item {\bf Compatibilidad de licencias}
Los paquetes de licencias de software que contienen explicaciones contradictorias, hacen imposible combinar el código de estos paquetes para crear nuevos paquetes de software. La compatibilidad de licencias entre una licencia copyleft y otra licencia es a menudo una compatibilidad de un solo sentido. La característica de esta compatibilidad es criticada por la Fundación Apache, que ofrece la ciencia permisiva Apache que no tiene esta característica. Las licencias que no son copyleft como la FOSS tienen menos problemas de compatibilidad. Por ejemplo si una licencia dice que las versiones modificadas deben mencionar a los desarrolladores en cualquier material de marketing, y otra dice que las versiones modificadas no pueden contener atribución adicional, entonces si alguien combina un paquete de software que usa una licencia con un paquete de software que usa la otra entonces sería imposible satisfacer la combinación porque estos requisitos son contradictorios y no se pueden completar a la vez, estos dos paquetes serían incompatibles. Cuando hablamos de las licencias de copyleft, ellas no son necesariamente compatibles con otras licencias de copyleft, incluso la versión 2 de la GPL por sí misma no es compatible con la versión 3 de la GPL.
	\item {\bf Propósito de uso}
Restricciones sobre el uso del software no son generalmente aceptadas de acuerdo con la FSF, OSI, Debian, o las distribuciones basadas en BSD. Algunos ejemplos son prohibir el software sólo para uso privado, para propósitos militares, para comparación o marco de mercado, para un uso que conlleve cuestiones éticamente cuestionables, o en organizaciones comerciales.
\end{itemize}
\newpage
\section {Copyleft, Copyright y Creative Commons}
\begin{itemize}
\item {\bf Copyleft}
Estrategia legal la cual permite hacer que el código sea libre, es decir, que pueda ser utilizado por cualquiera, modificarlo y mejorarlo para cualquier propósito, distribuir versiones originales y mejoradas con o sin lucro sin pedir permiso a nadie. Toda nueva versión o copia está basada en las mismas condiciones que el original.
\item {\bf Copyright}
Licencia más restrictiva, sólo el autor puede utilizarlo y si alguien quiere usarlo debe pedirle permiso al autor y pagarle. Si el software no tiene definida ningún tipo de licencia, adopta por defecto la copyright, cabe aclarar que su registro no es obligatorio. 
\item {\bf Creative Commons}
Determina cómo va a circular el software en internet, es decir, algunos autores pueden dar derechos a otras personas, son gratuitas y su registro no es obligatorio.
\end{itemize}
\newpage
\section {Tipos de Licencias}
El software no se vende, se licencia. Una licencia es aquella autorización formal con carácter contractual que un autor de un software da a un interesado para ejercer actos de explotación legales. Es decir, el software no se compra, sino que se adquieren una serie de derechos sobre el uso que se le puede dar. En las licencias de software libre esos derechos son muy abiertos y permisivos, apenas hay restricciones al uso de los programas. De ahí que ayude al desarrollo de la cultura. Pueden existir tantas licencias como acuerdos concretos se den entre el autor y el licenciatario. Desde el punto de vista del software libre, existen distintas variantes del concepto o grupos de licencias, las mismas pueden clasificarse en distintos grupos:\\

\begin{itemize}
\item Privativas:  Una licencia privativa concede, básicamente, el derecho a la instalación del software en el ordenador del usuario y su ejecución, de acuerdo con determinadas condiciones y restricciones, incumpliendo así, con algunas de las 4 libertades de un software libre.
\item Libres: estas licencias pueden ser clasificadas a su vez en 3 tipos distintos:
\begin{itemize}
\item Permisivas: es una licencia de software libre flexible respecto a la distribución, de modo que el software pueda ser redistribuido como software libre o software propietario, siendo libre la licencia original del autor.
La característica peculiar de estas licencias se debe a que no poseen copyleft, ya que el trabajo derivado no tiene por qué mantenerse con el mismo régimen de derechos de autor que el original. Esto maximiza la libertad para quien recibe el software y quiere desarrollar algo derivado, permitiéndole elegir entre el amplio abanico de licencias existentes. No obstante, desde el punto de vista de los usuarios, esto se puede considerar como una restricción a su libertad, en el sentido de que el software propietario siempre restringe las libertades de los usuarios del mismo y las licencias permisivas, abren esta posibilidad. Los ejemplos más conocidos de este tipo de licencias son la Licencia MIT y la Licencia BSD.
\item Copyleft débil: hace referencia a las licencias que no se heredan a todos los trabajos derivados, dependiendo a menudo de la manera en que estos se hayan derivado. Se utiliza generalmente para la creación de bibliotecas de software, con el fin de permitir que otros programas puedan enlazar con ellas y ser redistribuidos, sin el requerimiento legal de tener que hacerlo bajo la nueva licencia copyleft. Solamente se requiere distribuir los cambios sobre el software con copyleft débil, no los cambios sobre el software que enlaza con el. Un claro ejemplo de este tipo de licencias es la Licencia Pública de Mozilla.
\item Robustas: se centran en mantener la libertad a lo largo de toda la cadena de usuarios, desde su autor hasta el usuario final, para ello hace uso del copyleft obligando a que el trabajo derivado se mantenga con el mismo régimen de derechos de autor que el original. Así un software licenciado bajo GPL que sea modificado y distribuido por un eslabón intermedio de la cadena debe mantener la licencia GPL, maximizando la libertad de los usuarios finales.
La condición de distribuirse bajo la misma licencia original y al tratarse de software libre, implica que siempre se tenga acceso al código.
Los ejemplos más conocidos de este tipo de licencias son la GPL, LGPL y Affero GPL.
\end{itemize}
\end{itemize}
\newpage
\section{Papel de la FSF y OSI}

{\bf Licencias de código abierto aprobadas por el OSI}
El grupo OSI define y mantiene una lista de licencias de código abierto aprobadas por el mismo. OSI está de acuerdo con la FSF en la mayoría de las licencias de software libre, pero difiere en algunos de la lista de la FSF. Por ejemplo, se posiciona del lado de la definición de código abierto más que de la definición de software libre. Considera que las leyes del grupo licencias de software libre permisivas son una referencia de implementación de una licencia software libre. Esto es sus requerimientos para aprobar las licencias son diferentes.
\\
{\bf Licencias de software libre aprobadas por la FSF}
La FSF, el grupo que mantiene la definición de software libre, mantiene una lista de licencias de software libre no exhaustiva.12 La FSF es una organización sin ánimo de lucro cuya misión es promover la libertad de los usuarios de los ordenadores y defender los derechos de todos los usuarios del software libre. Los desarrolladores de software libre garantizan que todo el mundo tenga los mismos derechos para sus programas, cualquier usuario puede estudiar el código fuente, modificarlo, y compartir el programa. Por el contrario, la mayoría del software lleva una nota que impide a los usuarios estos derechos básicos, dejando a los usuarios susceptibles a los deseos de sus propietarios y a la vulnerabilidad de ser vigilados. La FSF prefiere las licencias de software libre copyleft (share-alike) mejor que las licencias de software libre permisivas para la mayoría de los propósitos. En su lista distingue entre las licencias de software libre que son compatibles o incompatibles con el la GPL de copyleft de la FSF.
\\
\newpage
\section{Garantìas, condiciones y limitaciones brindadas por las licencias}
\begin{itemize}
\item Garantìas: Son aquellas características del software que son permitidas o autorizadas por la licencia.
\begin{itemize}
\item Usar: Se garantiza el uso del software
\item Distribuir/Comerciar: Se permite su distribución y/o comercialización.
\item Modificar: Se permite modificar características y funcionalidades del programa.
\item Mejorar el programa y publicar las mejoras: Se pueden sugerir cambios y permitir que dichos cambios se lleven a cabo.
\item Derechos de patente para sus contribuidores: Las patentes dependen de las leyes nacionales respecto a las licencias del software.
\end{itemize}
\item Condiciones: Son aquellas características que debe poseer todo software que derive del poseedor de la licencia
\begin{itemize}
\item Misma licencia: Debe mantener su licencia cuando se realicen modificaciones.
\item Revelar fuente: Código fuente debe estar disponible cuando se distribuya el software
\item Informar de licencia y copyright: Se debe poseer una licencia e informar del copyright junto con el software
\item Red como medio de distribución: Se debe poder utilizar la red para recibir una copia del código fuente
\item Cambios documentados
\end{itemize}
\item Limitaciones: Son todas las características que no garantiza la licencia respecto al uso del software.
\begin{itemize}
\item Responsabilidad: No se hacen responsables por el uso dado al software
\item Garantía: No se provee ninguna garantía
\item Uso de marca registrada: No garantiza el uso de la marca registrada
\end{itemize}
\end{itemize}




\end{document}