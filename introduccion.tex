\chapter{Introducción al Software Libre}
\chapterauthor{Autor1, Autor2}

\section{Algo de Historia}

{\bf En 1970} la mayoría de las compañías tenían como forma de cortesía el hábito de publicar su código fuente en archivos legibles. Dichas mejoras eran publicadas, mejoradas por los usuarios y las empresas adoptaban estas mejoras.

AT\&T distribuía las primeras versiones de UNIX (Dennis Ritchie, Ken Thompson y Douglas McIlroy) sin costo para el gobierno y para los investigadores académicos, pero no se permitía su distribución ni la distribución de versiones modificadas. 
\\
{\bf Punto de debate: ¿Esto era Software Libre?}
{\it 

Concepto:
El término software libre refiere el conjunto de software (programa informático) que por elección manifiesta de su autor, puede ser copiado, estudiado, modificado, utilizado libremente con cualquier fin y redistribuido con o sin cambios o mejoras.
Como resultado de esto nace BSD, que luego deriva en SunOS, FreeBSD, NetBSD, PC-BSD, OpenBSD y Mac OS X.
}
\\
{\bf ¿Compartir o Robar?}

{\bf En 1976}, Bill Gates marcó el gran cambio de era cuando escribió, su ahora famosa Carta abierta a los aficionados, mandando el mensaje de que lo que los hackers llamaban «compartir» era, en sus palabras, «robar». Por ejemplo, en 1979, AT\&T empezó a hacer cumplir sus licencias restrictivas cuando la compañía decidió que podrían generar utilidades vendiendo el sistema Unix.9
\\
{\bf En 1979}, Caso, Reid vender Scribe a Unilogic. Se insertaron un conjunto de funciones dependientes del tiempo que desactivaba las versiones copiadas libremente después de un período de expiración de 90 días.
Reid había insertado una forma en que las compañías obligaban a los programadores a pagar por el acceso a la información.
\\
{\bf 1979 -1980} Primeros contactos con los acuerdos de no revelar (nondisclosure agreement) NDA, incidente en Carnegie Mellon con un ex investigador de Xerox.
\\
{\bf 1979 - 1980} La cultura del software cambia, empieza a nacer el código propietario.
El software se había convertido en una posesión tan valiosa y estratégica por lo que dejaron de publicar sus códigos fuentes.
\\
\\
{\bf Punto de Debate: ¿Que motivaron los proyectos de Software Libre?}
{\it 
\\
En 1983, Richard Stallman lanzó el proyecto GNU para escribir un sistema operativo completo, libre de restricciones para uso y modificación, pudiendo ser distribuido con o sin mejoras. Uno de los incidentes particulares que lo motivaron a esto fue el caso de una molesta impresora que no podía ser arreglada porque el código fuente no era revelado.
\\
Otro posible evento de inspiración para el proyecto GNU y su manifiesto fue el desacuerdo entre Stallman y Symbolics, Inc. sobre el acceso a las actualizaciones, por parte del MIT, que Symbolics había realizado a su máquina Lisp, la cual estaba basada en código del MIT.13 Poco tiempo después de su lanzamiento, acuñó el término "software libre" y para promover el concepto fundó la Free Software Foundation.
Una definición de software libre fue publicada en febrero de 1986.
}
\\
{\bf En 1985} Nace Free Software Foundation (Free Software Foundation América Latina. Fundada en el 2005)

Cuyo objetivo es promover y defender el uso y desarrollo de software libre, y el derecho de las personas a usar, estudiar, copiar, modificar y redistribuir software.

\section{ El Software Libre}

El Software Libre es un movimiento sociopolitico surgido hace más de dos décadas que tiene como objetivo promover la creación, expansión de software que respeta las libertades de los usuarios.

Refuerzo de definición: ¿Cuáles son esas libertades?

Existen por definición cuatro libertades básicas que debe poseer un software para ser considerado libre:

{\bf Libertad 0:}
La libertad de usar el software de la forma que uno quiera. De forma opuesta, muchos programas de computadora privativos no permiten ciertos usos como por ejemplo, crear otro producto que compita con ese software.
\\
{\bf Libertad 1:}
La libertad de estudiar y mejorar el software. Esto es imposible si no tenemos acceso al código fuente, por eso el Software Libre se caracteriza por distribuirlo además del binario/ejecutable.
\\
{\bf Libertad 2:}
La libertad de copiarlo y compartirlo. El Software Libre es, como ya dije, un movimiento social. Cree que el compartir es algo bueno y no es un delito. Se han inventado conceptos descabellados como “piratería” para defenestrar el maravilloso acto de ser una persona generosa, una buena persona. El Software Libre defiende esos valores y por eso no es ilegal copiarlo y compartirlo. Es más, su éxito se basa justamente en eso.
\\
{\bf Libertad 3:}
La libertad de distribuir las mejoras hechas. Gracias a la Libertad 1 podemos aprender como funciona el software, esta libertad nos permite hacer nuestro aporte. Esta libertad es la responsable de generar comunidades de personas que mejoran y comparten sus soluciones unos con otros.
\\
{\bf En 1989} Nace la licencia GPL, creada por Richard Stallman en 1989 para proteger los programas liberados como parte del proyecto GNU.
\\
\\
{\bf Punto de debate ¿SW Libre solo para algunos? O ¿Libre de compromiso de calidad?}
{\it 
\\
Existen algunas desventajas, como en cualquier tipo de software, como por ejemplo:
El software libre se compra o se adquiere sin garantías explícitas del fabricante o autor.
Las modificaciones o problemas encontrados requiere de la dedicación de recursos institucionales, así como a la adaptación a las necesidades del usuario que lo utilice.
Los usuarios deben tener unos conocimientos mínimos de programación para modificarlo y adaptarlo al contexto educativo.
}
\\
{\bf Punto de debate ¿No compartir codigo es malo? }
{\it 
\\
La Free Software Foundation argumenta que el software es conocimiento y debe poderse difundir sin trabas. Su ocultación es una actitud antisocial y la posibilidad de modificar programas es una forma de libertad de expresión, aunque sin olvidar una estructura jerarquizada por la meritocracia.
\\
La motivación pragmática, defendida por la Open Source Initiative, argumenta ventajas técnicas y económicas, con respecto a evitar una tragedia de los anticomunes mejorando los incentivos.
Aparte de estas motivaciones, quienes trabajan con software libre suelen hacerlo por muchas otras razones, que van desde la diversión a la mera retribución económica, que es posible debido a modelos de negocio sustentables.
\\
La mayoría de los desarrolladores de software libre provienen de países industrializados. El Mapa de desarrolladores del proyecto Debian muestra que la mayoría de desarrolladores se encuentran en Europa y Estados Unidos.
Los usuarios deben estar al corriente de las modificaciones que se les haga al software para evitar confusiones.
}
\\
{\bf En 1991} Nace el kernel Linux bajo  GPL creado por Linus Torvalds.
\\
{\bf En1998} Nace La Open Source Initiative (OSI)
\\
{\bf En 2001} Corporación Microsoft enfatiza su postura en contra de la Licencia Pública General de GNU.
Categorizando a GPL como una licencia "viral"
\\
{\bf En 2001} Conferencia de Stallman masiva en New York en respuesta a los dichos de MicroSoft.
\\
{\bf Punto de Debate: ¿Como afectan los distintos pensamientos a la industria del SoftWare? }
{\it 
\\
\emph{Para el sector del software privativo:}
GNU socava fundamentalmente el sector de software comercial independiente debido a que efectivamente hace que sea imposible el distribuir software bajo una base donde los compradores pagan por el producto algo más que tan sólo el costo de la distribución y limita el presupuesto para la investigación de nuevas tecnologias.
\\
\emph{Para los seguidores del SL, el sector privado:}
Lleva al abuso del monopolio y al estancamiento, fuertes compañías se absorben todo el oxígeno del mercado quitándoselo a los competidores rivales y a los principiantes innovadores.
}
\\
{\bf Convivencia de ambos pensamientos.}
El éxito mutuo de GNU/Linux, el sistema operativo amalgamado construido alrededor del núcleo
Linux protegido por la GPL, y de Windows a lo largo de los últimos 10 años revela la sabiduría de
ambas perspectivas.
\\
{\bf Punto de Debate: ¿Cual es el grado de control de nuestros software?}
\\
Si está usando alguna versión del sistema operativo Microsoft Windows usted está cediendo el control de toda su información, quedando en manos de alguien más. No tiene forma de saber que está haciendo su computadora en este instante.
\\
Y; aunque fuera  técnico tampoco lo sabría. El propietario del software es quien lo sabe y usted no lo es. 
Él decidió negarle a el control y claro, usted aceptó siendo tan solo un consumidor.

\section{Categorías de software libre y software que no es libre}
	.
\begin{description}
	\item[$\cdot$ Software libre]
	\item[$\cdot$ Software de código abierto («Open Source»)]
	\item[$\cdot$ Software de dominio público]
	\item[$\cdot$ Software con copyleft]
	\item[$\cdot$ Software libre sin copyleft]
	\item[$\cdot$ Software con licencia permisiva, laxa]
	\item[$\cdot$ Software con licencia GPL]
	\item[$\cdot$ Software GNU bajo copyright de la FSF]
	\item[$\cdot$ Software que no es libre]
	\item[$\cdot$ Software Privativo]
\end{description}

Leer mas en: https://www.gnu.org/philosophy/categories.es.html
\\
{\it
\emph{Nota al Margen:} Con el SL se ganan dos cosas: una amistad incrementada y la habilidad para pedir prestado interesantes recetas como intercambio.
\\
\emph{Paradoja de SL En China:}
Se ha demorado mucho en ser entendido. Hacer la comparación entre software libre y libertad de expresión es más difícil cuando no hay libertad de expresión
}
\section{Algunos Hitos del SL}

\begin{description}
\item[$\cdot$ 1969 – Creación de ARPANET]
\item[$\cdot$ 1970 – Creación de UNIX en los Laboratorios Bell]
\item[$\cdot$ 1979 – Creación de BSD en la Universidad de California]
\item[$\cdot$ 1983 – Richard Stallman inicia el movimiento GNU]
\item[$\cdot$ 1984 – Se crea el entorno gráfico X Window System]
\item[$\cdot$ 1985 – Se funda la Free Software Fundation]
\item[$\cdot$ 1989 – GNU está casi completo, pero falta el núcleo]
\item[$\cdot$ 1991 – Se inicia el desarrollo del núcleo Linux por Linus Torvalds]
\item[$\cdot$ 1992 – Primera distribución GNU/Linux: SLS]
\item[$\cdot$ 1993 – Creación de las distribuciones Debian y Slackware]
\item[$\cdot$ 1994 – Creación de Redhat y Suse]
\item[$\cdot$ 1996 – Inicio del proyecto KDE]
\item[$\cdot$ 1997 – Inicio del movimiento Open Source. Creación de la OSI. Inicio del proyecto GNOME]
\item[$\cdot$ 1998 – Netscape es liberado como software libre]
\item[$\cdot$ 2002 – Creación de Gentoo y Linex]
\item[$\cdot$ 2003 – Creación de Knoppix y Fedora Core]
\item[$\cdot$ 2004 – Creación de Ubuntu]
\item[$\cdot$ 2005 - Git creado por Linus Torvalds]
\end{description}


\section{Aspectos Históricos del Software Libre en América Latina}

América Latina y el Caribe son regiones consumidoras de tecnología, ya sea importada de otras partes del mundo o producida localmente por sucursales de compañías extranjeras. A partir de los procesos de privatización y desregularización de las telecomunicaciones, los servicios también están dominados por gigantes globales. La mayoría del software propietario líder en el mercado ha sido traducido a Español y Portugués en busca de un creciente mercado de software que crece un 18\% anual.
Algunos estudios señalan que del total de usuarios de Linux en el mundo, aproximadamente un 5\% se encuentra concentrado en países de América Latina, específicamente en Brasil, México, Chile y Argentina.
\\
\\
{\bf Proyectos.}
\\
\\
{\bf Asociación Software Libre Argentina (SoLAr)}
\\
SoLAr es una organización que agrupa a usuarios y desarrolladores de Software Libre de Argentina, con el objetivo de generar un espacio de representación y promover las ventajas tecnológicas/sociales/políticas del Software Libre.
Es un grupo interdisciplinario de personas que busca conformar una Asociación Civil homónima, con el objetivo de promover las ventajas tecnológicas, sociales y políticas del Sofware Libre, creando un espacio orgánico de representación de los individuos y comunidades. El grupo SoLAr, está conformado por personas de diversos lugares de Argentina (como Corrientes, Buenos Aires, Jujuy, Salta, Carlos Casares, La Plata, Trevelín, Rosario) y diferentes perfiles (usuarios, desarrolladores, ingenieros, matemáticos,
contadores, antropólogos), en busca de un espacio de articulación enriquecido por la diversidad. So-LAr está dando sus primeros pasos en su conformación como sociedad y eso requiere de la colaboración de todos aquellos interesados en el Software Libre. Todos tienen la posibilidad de tener una participación activa y comprometida; por eso SoLAr está generando un espacio.
\\
Algunos de los objetivos actuales de SoLAr son:

\begin{description}
\item $\cdot$ Trabajar para que se garanticen los derechos humanos y libertades fundamentales en relación al software.
\item $\cdot$ Trabaja a nivel nacional en la República Argentina, y coopera en sus respectivas tareas y ámbitos con: las organizaciones locales de Argentina, las organizaciones nacionales similares de otros países y las organizaciones internacionales.
\item $\cdot$ Solar no hace distinción de ninguna clase sea de género, religión, raza, ocupación laboral o comercial, pertenencia a grupos, organizaciones o instituciones, partidaria, u otra, más allá de sus objetivos -obviamente políticos e ideológicos- en relación al Software.
\item $\cdot$ Solar no fija posición oficial en relación a diferentes alternativas tecnológicas dentro del Software Libre, GNU/Linux/BSD/Otros, en tanto sean Software Libre, son promovidos por Solar.
\item $\cdot$ Solar trabajará para eliminar cualquier miedo, incertidumbre y duda injustificadas con relación al Software Libre y se opondrá sistemáticamente a todas las campañas que se realicen con el fin de difundir estas especies.
\item $\cdot$ Solar promoverá, alentará e incentivará el uso de software que brinde libertades a las personas.
\item $\cdot$ Se opone al otorgamiento de patentes de software, que afectaran no solamente al Software Libre sino también y fundamentalmente a los pequeños y medianos desarrolladores de software privativo.
\item $\cdot$ Para terminar con la Brecha Digital, hay que acabar con la pobreza y las diferentes formas de exclusión.
\end{description}
Para más información sobre la asociación SoLAr véase Proyecto SoLAr [http://www.solar.org.ar]
\\
\\
{\bf Projeto Software Livre Brasil (PSL-Brasil)}
\\
El Projeto Software Livre Brasil [http://www.softwarelivre.org] es una iniciativa no gubernamental que reúne instituciones publicas y privadas de Brasil: poder publico, universidades, empresarios, grupos de usuarios, hackers, ONG's. Su principal objetivo es la promoción del uso y del desarrollo de Software Libre como una alternativa de libertad de expresión, económica y tecnológica. Estimulando el uso del Software Libre, como proyecto de investigación y enriquecimiento del conocimiento local a partir de un nuevo paradigma de desarrollo sustentado y de una nueva postura, que incide en la cuestión tecnológica como contexto de construcción de un mundo con inclusión social e igualdad de acceso a los avances tecnológicos.
\\
El proyecto esta repartido en varios estados de Brasil, a través de PSL's Estaduais (Proyecto Software Libre en Estados), que son partes integrantes de PSL-Brasil. También esta repartido por temas como PSL-mujeres, PSL-Jurídico e PSL-empresas.
\\
\\
{\bf Fundación Software Libre A. C. (México)}
\\
Fundación de Software Libre A. C. [http://www.fsl.org.mx/tiki-index.php] es una asociación civil dedicada a promover, difundir y apoyar el uso y desarrollo del Software Libre. Se trata de una entidad legalmente constituida para dar un apoyo legal y de estructura al movimiento de Software Libre en México, es un proyecto a tiempo indefinido; por razones prácticas y estratégicas su domicilio se localiza en la Ciudad de Toluca, Estado de México; aún cuando a su debido tiempo se establezcan oficinas o representaciones en los demás estados de la República, del Distrito Federal o del Extranjero.
\\
Algunos de los objetivos principales eran:

\begin{description}
\item $\cdot$ Conformarse como un organismo consultor que ofrezca directrices para los proyectos de tecnología que requiera la región.
\item $\cdot$ Volverse un integrador de proyectos que convoque a las diferentes personas y empresas implicadas en el desarrollo del Software Libre para la realización de los mismos.
\item $\cdot$ Buscar mecanismos de patrocinio e inversión para ayudar a los gobiernos, instituciones, empresas y gente a realizar los proyectos de tecnología que México y Latino América requieran.
\item $\cdot$ Ser un promotor garante del movimiento de Software Libre. Proporcionar servicios de Certificación en temas relacionados con el Software Libre en México y Latino América.
\item $\cdot$ Conformar una Cámara de representación comercial que contenga un registro de gente, empresas e instituciones que utilizan o desarrollan Software Libre en México y Latino América.
\end{description}

Sin embargo pasaron los meses y no parecía verse alguna acción concreta de avance del proyecto.
\\
\\
{\bf Software Libre (Chile)}
\\
Software Libre de Chile es un proyecto infocomunicacional que promueve los principios del movimiento GNU. Un centro de conocimiento acerca del avance del Movimiento del Software Libre en Chile y el ámbito global bajo una dinámica de construcción colaborativa.
\\
El Software Libre tuvo sus comienzos en Chile entre 1987 y 1991 cuando se crean las condiciones para la conexión nacional a Internet y se establecen en el Departamento de Informática de la Universidad de Chile los primeros repositorios FTP56 (File Transfer Protocol) con copias de los programas del Proyecto GNU. Estas aplicaciones empezaron a difundirse entre los centros de informática, siendo utilizadas como complementos de los sistemas operativos UNIX que gestionaban los recursos telemáticos en las universidades.
\\
No es aventurado señalar que Chile no es hasta el momento un modelo de referencia en relación con el activismo social del Software Libre. Sin embargo, han surgido en los últimos años una serie de actores que están dinamizando poderosamente el movimiento en Chile, operando como catalizadores de la apropiación tecnológica en la sociedad civil.

\section{Proyectos relacionados con Linux en America Latina}

{\bf Kernel 2.4 de Linux}
\\
Marcelo Tosatti se convirtió en el mantenedor de la versión 2.4 del kernel de Linux con solo 18 años, en noviembre de 2001. Su primera fue la 2.4.16 el 26 de noviembre de ese mismo año. Dos años más tarde realizo la versión 2.4.23 y planeo pronto poner la versión estable del kernel 2.4 en modo mantenimiento, dirigiendo solo los fallos y problemas de seguridad.
\\
\\
{\bf Proyecto Gnome}
\\
El proyecto fue iniciado por los programadores de software libre mexicanos Miguel de Icaza y Federico Mena y forma parte oficial del proyecto GNU. Nació como una alternativa a KDE bajo el nombre de GNU Network Object Model Environment (Entorno de Modelo de Objeto de Red GNU). Actualmente, incluyendo al español, se encuentra disponible en 166 idiomas.
\\
Tiene como objetivo la libertad para crear un entorno de escritorio que siempre tendrá el código fuente disponible para reutilizarse bajo una licencia de software libre.

\section{Futuro del Software Libre}


	El SL a tenido ha sido un agente de cambio y de evolución en la industria del software.
Muchas empresas se han provistos de sus desarrollos para su beneficio propio y en algunos casos incumpliendo con las licencias GPL al no publicar sus códigos fuentes.
	Recientemente han empezado haber juicios a favor de FSF sobre estas prácticas que afirman y renuevan el SL.

Leer mas en: http://www.eveliux.com/mx/El-futuro-del-Software-Libre.html

\section{Ética y el software libre}

{\bf Caso Android. ¿ Android es Software Libre?}
\\
\\
Android. Android es un \emph{software de código abierto} que se distribuye bajo la licencia Apache V2.
\\
La licencia Apache V2 da permiso para:
\\
Utilizar el software para cualquier propósito, distribuirlo, modificarlo y distribuir las modificaciones
no tiene copyleft por lo que las versiones modificadas no tienen que ser distribuidas como software libre.
Compatible con GPL3 pero no con las anteriores.
Incluye provisiones de protección respecto a patente.
\\
Android es un software de código abierto con  acceso al código fuente del mismo, pero terceras partes, como por ejemplo fabricantes, pueden modificarlo o añadir código y este no tiene porqué ser liberado. 
\\
Leer mas en: https://www.androidsis.com/android-es-software-libre-estamos-en-lo-cierto/

\section{Conclusion}
	
Gracias a la cultura de compartir el conocimiento (El software) la industria al dia de hoy se vio afectada favorablemente, derivando en importantes desarrollos como lo fueron Unix, BSD, TCP/IP, etc.
Cuando las empresas en su afán de consolidar su posición en el mercado, entendieron que el software era un activo valioso decidieron que esta no debía ser compartido tan a la ligera, dando lugar a las NDA y al software privativo.
\\
En este movimiento se han cometido muchos abusos, hasta patentar y registrar cosas triviales que permiten a las empresas demandar a otras por un Doble Click, ¡literalmente! (Registro de Microsoft).
\\
Ya hace tiempo, las empresas también entendieron  que no solo el software es un activo importantes, sino, aún más valioso son los datos generados por estos. Dando lugar a una proliferación de software gratuito o de código abierto con el objetivo de obtener datos de lo más variado y estos ser usados con fines estratégicos y/o comerciales.
\\
Cuando no tenemos el código fuente no es posible saber como este esta hecho, tampoco como este funciona, ya que no es transparente. 
Solo si sabemos como funciona el software podemos hablar de transparencia y solo cuando tenemos acceso a modificar el funcionamiento del software estamos habilitados a dejar ser simples consumidores.
\\
La distinción crucial del software es justamente ésta: el acceso a poder entender cómo funciona y poder modificarlo. 
\\
Como usuarios debemos entender cómo estas empresas actúan, aceptando o rechazando sus prácticas.
\\
Otra observación sobre las grandes corporaciones y su software privativo es el empoderamiento a través de estrategias destinadas a copar el mercado con sus producto y estableciendolos como protocolos defactos.
El usuario domestico solo es una arista menor, siendo su principal interes las empresas que deberan adaptarse a las tecnologias dispuesta en el mercado, que mediane la renovación constante de licencias y actulizaciones generan la depencia buscada.
\\
Si todos usan una misma tecnologia, esta se vuelve el standar, provocando un efecto contagio.
\\
\emph{El software una vez creado, es facil de copiar, no puede ser escaso, por lo que su precio deberia tender a 0}, la unica manera de hacer que la demanda se mantenga, es mediante la creacion de restricciones artificiales sobre el software, como por ejemplo: Licencias de uso por tiempo determinado.


